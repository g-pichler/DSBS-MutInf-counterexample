\synctex=1

\documentclass[a4paper,12pt]{scrartcl}

%%%%%%%%%%%%%%%%% 
% Customization %
%%%%%%%%%%%%%%%%% 

% Load all packages
\usepackage[utf8]{inputenc}
\usepackage[T1]{fontenc}
\usepackage{amsmath}
\usepackage{geometry}
\usepackage{amsfonts}
\usepackage{amsthm}
\usepackage{mathtools}
\usepackage{biblatex}
\usepackage{setspace}
\usepackage{hyperref}
\usepackage[capitalize]{cleveref}
\usepackage{listings}

\newtheorem{lemma}{Lemma}
\addbibresource{lit.bib}

\renewcommand{\ttdefault}{pcr}

%%%%%%%%%%%%%%%%%%%%%%%%%%%%%%%%%%%%%
% Definition for this document only %
%%%%%%%%%%%%%%%%%%%%%%%%%%%%%%%%%%%%%
\newcommand{\heps}{\ensuremath{\hat \varepsilon}}
\newcommand{\cvx}[1]{\ensuremath{\mathrm{conv}(#1)}}
\newcommand{\hPcond}[2]{\ensuremath{\hat{\mathrm{P}}\{#1|#2\}}}
\newcommand{\Pcond}[2]{\ensuremath{\mathrm{P}}\{#1|#2\}}
\newcommand{\hProb}{\ensuremath{\hat{\mathrm{P}}}}
\newcommand{\Prob}{\ensuremath{\mathrm{P}}}
\newcommand{\ul}[1]{\ensuremath{\underline{#1}}}
\newcommand{\wt}[1]{\ensuremath{\widetilde{#1}}}
\newcommand{\mat}[1]{\ensuremath{\mathbf{#1}}}
\newcommand{\vt}[1]{\ensuremath{\mathbf{#1}}}
\newcommand{\rv}[1]{\ensuremath{\mathsf{\uppercase{#1}}}}
\newcommand{\hrv}[1]{\ensuremath{\hat{\rv{#1}}}}
\newcommand{\transp}{\ensuremath{^{\mathrm T}}}
\newcommand{\AAA}{\ensuremath{\mathcal A}}
\newcommand{\HHH}{\ensuremath{\mathcal H}}
\newcommand{\SSS}{\ensuremath{\mathcal S}}
\newcommand{\OOO}{\ensuremath{\mathcal O}}
\newcommand{\bernoulli}[1]{\ensuremath{\mathcal{B}(#1)}}
\newcommand{\DSBS}[1]{\ensuremath{\mathrm{DSBS}(#1)}}
\newcommand{\RR}{\ensuremath{\mathbb R}}
\newcommand{\defas}{\vcentcolon=}

\def\barcirc{\mathrel{\barcirci}}
\def\barcirci{{%
    \setbox0\hbox{\ensuremath{\relbar\!\!\relbar}}%
    \rlap{\hbox to \wd0{\hss\ensuremath{\circ}\hss}}\box0
}}
\newcommand{\mkv}{\ensuremath{\barcirc}}

\newcommand{\binEnt}[1]{\ensuremath{\mathrm{H}(#1)}}
\newcommand{\binEntInv}[1]{\ensuremath{\mathrm{H}^{-1}(#1)}}
\newcommand{\mutInf}[2]{\ensuremath{\mathrm{I}(#1;#2)}}
\newcommand{\Nto}[1]{\ensuremath{\{1,2,\dots,#1\}}}

\newtheorem{theorem}{Theorem}

\lstloadlanguages{Octave}
\lstset{%
  language=Octave,
  numbers=left,
  stepnumber=2,
  numberstyle=\tiny,
  basicstyle=\footnotesize\ttfamily,
  keywordstyle=\bfseries,
  morekeywords={inf,sup,convhulln,precedes,infsupdec},
  breaklines=true,
}

% This makes sure that firstnumber = firstline
\makeatletter
\patchcmd{\lst@GLI@}% <command>
  {\def\lst@firstline{#1\relax}}% <search>
  {\def\lst@firstline{#1\relax}\def\lst@firstnumber{#1\relax}}% <replace>
  {\typeout{listings firstnumber=firstline}}% <success>
  {\typeout{listings firstnumber not set}}% <failure>
\makeatother

%%%%%%%%%%
% Layout %
%%%%%%%%%%
\geometry{a4paper, inner=2.5cm, outer=2.5cm, top=3.25cm, bottom=3.25cm, headsep=1cm}
\setstretch{1.2}
\setlength\parskip{0pt
	plus .1\baselineskip
}
\setlength\topsep{.5\baselineskip
	plus .1\baselineskip
}

\sloppy
\allowdisplaybreaks
\raggedbottom

%%%%%%%%%%%%%%%%%%%%
% Document Details %
%%%%%%%%%%%%%%%%%%%%
\title{Documentation}
\author{Georg Pichler}
\date{\today}


\begin{document}

% What would you like?
%\longversion %default
%\shortversion %or just put "shortversion" somewhere in the filename

% Title Page
\maketitle

\section{Overview}
\label{sec:overview}
Assume $p \in (0,\frac 12)$ and define the set $\SSS' \subseteq \RR^3$ as
\begin{align}
  \SSS' \defas \smash{\bigcup_{0 \le \alpha, \beta \le \frac 12}} \big\{
  (\mu, R_1, R_2) :
  R_1 &\ge 1 - \binEnt{\alpha},\, \nonumber\\*
  R_2 &\ge 1 - \binEnt{\beta},\, \nonumber\\*
  \mu &\le 1 - \binEnt{\alpha * p * \beta} 
        \big\},
\end{align}
where $\binEnt{t} \defas -t\log_2(t) - (1-t) \log_2(1-t)$ is the binary entropy function and $a * b \defas a(1-b)+(1-a)b$ is the binary convolution.
Let $\SSS \defas \cvx{\SSS'}$ denote the convex hull of $\SSS'$.

Note that we can also define $\SSS'' \subseteq \RR^3$ as
\begin{align}
  \SSS'' \defas \smash{\bigcup_{0 \le \alpha, \beta \le \frac 12}} \big\{
  (\mu, R_1, R_2) :
  R_1 &= 1 - \binEnt{\alpha},\, \nonumber\\*
  R_2 &= 1 - \binEnt{\beta},\, \nonumber\\*
  \mu &= 1 - \binEnt{\alpha * p * \beta} 
        \big\}
\end{align}
and obtain $\SSS' = \SSS'' + \OOO$, where $\OOO \defas \RR_- \times \RR_+ \times \RR_+ $. Letting $\wt\SSS \defas \cvx{\SSS''}$, we have $\SSS = \wt\SSS + \OOO$.

The random variables $(\rv x, \rv y) \sim \DSBS{p}$ are a doubly symmetric binary source with parameter $p$, i.e., $\rv x \sim \bernoulli{\frac12}$ is a Bernoulli random variable with parameter $\frac12$ and $\rv y \defas \rv x \oplus \rv z$, where $\oplus$ denotes the binary ``xor'' operation. The random variable $\rv z \sim \bernoulli{p}$ is independent of $\rv x$.

We want to show the following two statements:
\begin{theorem}
  \label{thm:1}
  For $p=0$, there exist random variables $(\rv u, \rv v)$ with\footnote{For random variables $\rv x$, $\rv y$ and $\rv z$ we use the notation $\rv x \mkv \rv y \mkv \rv z$ to denote that $\rv x$, $\rv y$ and $\rv z$ form a Markov chain in this order, i.e, $\rv x$ is independent of $\rv z$ given $\rv y$.} $\rv u \mkv \rv x \mkv \rv z \mkv \rv v$ such that $(\mu, R_1, R_2) \notin \SSS'$, where
  \begin{align}
    R_1 &= \mutInf{\rv u}{\rv x} \label{eq:1:1} \\
    R_2 &= \mutInf{\rv v}{\rv z} \label{eq:1:2} \\
    \mu &= \mutInf{\rv u}{\rv v} \label{eq:1:3} .
  \end{align}
\end{theorem}

\begin{theorem}
  \label{thm:2}
  For $p=0.1$, there exist random variables $(\rv u, \rv v)$ with $\rv u \mkv \rv x \mkv \rv z$ and $\rv x \mkv \rv z \mkv \rv v$, such that $(\mu, R_1, R_2) \notin \SSS$, where
  \begin{align}
    R_1 &= \mutInf{\rv u}{\rv x} \\
    R_2 &= \mutInf{\rv v}{\rv z} \\
    \mu &= \mutInf{\rv u}{\rv x} + \mutInf{\rv v}{\rv z} - \mutInf{\rv u \rv v}{\rv x \rv z} .
  \end{align}
\end{theorem}

\section{Proof of Theorem~\ref{thm:1}}
\label{sec:proof-1}

For $a \in [0,1]$ let $(\rv u, \rv v)$ satisfy $\rv u \mkv \rv x \mkv \rv z \mkv \rv v$ as well as
\begin{align}
  \textrm{p}_{\rv u | \rv x}(u | x) =
  \begin{cases}
    a & u=x=0 \\
    0 & u=0, x=1 \\
    1-a & u=1, x=0 \\
    1 & u=x=1
  \end{cases} ,
      &&
         \textrm{p}_{\rv v | \rv z}(v | z) =
         \begin{cases}
           0 & v=z=0 \\
           a & v=0, z=1 \\
           1 & v=1, z=0 \\
           1-a & v=z=1
         \end{cases} .
\end{align}
We obtain $R \defas R_1 = R_2 = \binEnt{\frac{a}{2}} - \frac{1}{2}\binEnt{a}$ and $\mu = \mutInf{\rv u}{\rv v} = 2R - a$ from \cref{eq:1:1,eq:1:2,eq:1:3}. Note that $\rv x = \rv z$ as $p = 0$.
We fix $a = 0.8$ and use interval arithmetic\footnote{We will use underlined letters to denote intervals and use the following notation to compare intervals: $\ul a < \ul b$ iff $a < b$ for all $a \in \ul a$ and $b \in \ul b$. Additionally, we define $b < \ul a$ iff $\{b\} < \ul a$.}~\cite{Moore2009Introduction} to compute compute intervals $\ul R$ and $\ul\alpha$ such that $R \in \ul R$ and $\ul R \subseteq 1-\binEnt{\ul\alpha}$.
We then compute $\ul\mu_{\text{b}} \defas 1 - \binEnt{\ul\alpha * \ul\alpha}$ and finish the poof by showing that $\ul\mu \defas 2\ul R - a > \ul\mu_{\text{b}}$.

The computation is done in \texttt{counterexample1.m}:
\lstinputlisting[%
firstline=17,
lastline=33,
]{../counterexample1.m}

Evaluation confirms that there is a gap with $\ul\mu - \ul\mu_{\text{b}} > 10^{-3}$:

\texttt{Gap is at least 7.948583e-03}

\section{Proof of Theorem~\ref{thm:2}}
\label{sec:proof-2}

\subsection{The Counterexample}
\label{sec:counterexample}

By random search we numerically found an approximate distribution $\hPcond{\hrv u=u, \hrv v=v}{\rv x=x, \rv z=z}$ with $u,v,x,z \in \{0,1\}$, that will serve as a candidate.
We interpret the conditional distribution $\hProb$ as a vector in $[0,1]^{16}$. Additionally $\hProb$ has the property that $\hPcond{\hrv u=u}{\rv x=x} \approx \hPcond{\hrv v=u}{\rv z=x}$, up to numerical error.
Notice that the required Markov chains $\rv u \mkv \rv x \mkv \rv z$ and $\rv x \mkv \rv z \mkv \rv v$, as well as the requirements $\sum_{u,v} \Pcond{\rv u=u, \rv v=v}{\rv x=x, \rv z=z} = 1$ for all $x,z \in \{0,1\}$ are linear constraints on $\Pcond{\rv u=u, \rv v=v}{\rv x=x, \rv z=z} \in [0,1]^{16}$.
By calculating the kernel of the corresponding matrix, we can find a symbolic approximation $\mathrm{P}$ of $\hat{\mathrm{P}}$ which is a probability distribution satisfying the required Markov chains, as well as $\Pcond{\rv u=u}{\rv x=x} = \Pcond{\rv v=u}{\rv z=x}$ for $u,x \in \{0,1\}$.
Hence, we found the candidate $(\mu^\circ, R^\circ, R^\circ)$ with $R^\circ=\mutInf{\rv u}{\rv x}=\mutInf{\rv v}{\rv z}$ and $\mu^\circ = 2R^\circ - \mutInf{\rv u \rv v}{\rv x \rv z}$. Using interval arithmetic\footnote{We will use underlined letters to denote intervals and use the following notation to compare intervals: $\ul a < \ul b$ iff $a < b$ for all $a \in \ul a$ and $b \in \ul b$. Additionally, we define $b < \ul a$ iff $\{b\} < \ul a$.}~\cite{Moore2009Introduction}, we compute intervals $\ul\mu^\circ \ni \mu^\circ$ and $\ul R^\circ \ni R^\circ$.

\subsection{The Lower Bound}
\label{sec:lower-bound}

Let $\hat\mu(R_1,R_2) \defas \max\{\mu : (\mu, R_1, R_2) \in \SSS \}$ and by slightly abusing notation $\hat\mu(R) \defas \hat\mu(R,R)$. We will upper bound $\hat\mu(R^\circ)$ in order to ultimately prove $\hat\mu(R^\circ) < \ul\mu^\circ$.

Given a rate pair $(R_1, R_2) \in [0,1]^2$, we can obtain intervals $\ul\alpha, \ul\beta$ such that $R_1 \in 1 - \binEnt{\ul\alpha}$ and $R_2 \in 1 - \binEnt{\ul\beta}$ using an interval implementation of the inverse binary entropy function. This yields an interval $\ul\mu(\ul\alpha, \ul\beta) \supseteq 1 - \binEnt{\ul\alpha * p * \ul\beta}$.
Given strictly increasing rates $R(i) \in [0,1]$ with $i \in \Nto{K}$ and $0 = R(1) < R(2) < \dots < R(K) = 1$,
we perform this process on the $K \times K$ grid $(R(i), R(j))_{i,j=1\dots K}$ where $K \approx 800$ and obtain $\ul\mu(i,j)_{i,j=1\dots K}$, an interval representation of $\SSS''$. However, we are interested in $\wt\SSS = \cvx{\SSS''}$.
We shift $\ul\mu$ by one and obtain $\ul{\wt\mu}(i,j) \defas \ul\mu\big(\min(i+1,K),\min(j+1,K)\big)$.

\begin{lemma}
  \label{lem:bound}
  Let $\vt a \in \OOO$, $b \in \RR$. If $\vt a \cdot (\ul{\wt\mu}(i,j), R(i), R(j)) \ge b$ for all $i,j = 1\dots K$, then $\vt a \cdot \vt x \ge b$ for all $\vt x \in \wt\SSS$.   
\end{lemma}
\begin{proof}
  % Observe that $\mu(i,j)$ is increasing in the sense that $\mu(i,j) \le \mu(i+1,j)$ and $\mu(i,j) \le \mu(i,j+1)$.
  If there is a point in $\vt x \in \wt\SSS$ with $\vt a \cdot \vt x < b$, then there must also be a point $\vt x' = (\mu, R_1, R_2)  \in \SSS''$ with $\vt a \cdot \vt x' < b$. There is a square $[R(i), R(i+1)] \times [R(j), R(j+1)] \ni (R_1, R_2)$. Clearly $\mu \not\ge \ul{\wt\mu}(i,j)$ and hence $\vt a \cdot (\ul{\wt\mu}(i,j), R(i), R(j)) \not\ge \vt a \cdot \vt x' < b$.
\end{proof}

We obtain $\vt a = (a_1, a_2, a_3) \in \OOO$, $b \in \RR$ and verify the condition of \cref{lem:bound}. Clearly $\vt a \cdot \vt x \ge b$ for all $\vt x \in \SSS$. It remains to verify that
\begin{align}
  a_1 \ul\mu^\circ + (a_2 + a_3) \ul R^\circ < b  .
\end{align}

\subsection{Verification}
\label{sec:verification}

The verification of the counterexample is performed by \texttt{counterexample2.m}. After some preliminary checks and loading the required packages, the approximate distribution $\hPcond{\hrv u=u, \hrv v=v}{\rv x=x, \rv z=z}$ is imported from \texttt{numeric\_distribution.dat} and we set some parameters that will be used for calculating the sequence of rates $0 = R(1) < R(2) < \dots < R(K) = 1$. We also initialize the inverse binary entropy function class \texttt{ibinent}.

\lstinputlisting[%
firstline=14,
lastline=22,
]{../counterexample2.m}

The meaning of $K$, $Q$, $\mathrm{bdp}$ is a follows. The first $\frac K4$ rates will be spaced uniformly in $[0, \mathrm{bdp}]$. The remaining $\frac{3K}{4}$ rates are calculated as $R=1-x^Q$, where $x$ is uniformly spaced in $[0, (1-\mathrm{bdp})^{\frac 1Q})$.

Next we construct the symbolic distribution $\mathrm{P} = \texttt{Psc\_UVXZ}$. We do so by first constructing a symbolic distribution $\texttt{Psv\_XZ} = \Prob(\rv x=x, \rv z=z)$ and the symbolic reference distribution $\texttt{Psv\_UVgXZ\_ref} = \mathrm{P}_{\mathrm{ref}}(\hrv u=u, \hrv v=v|\rv x=x, \rv z=z)$ that satisfies the long Markov chain $\hrv u \mkv \rv x \mkv \rv z \mkv \hrv v$ and has two identical BSCs $\rv x \to \hrv u$, $\rv z \to \hrv v$.
Then we apply the projection matrix returned by \texttt{get\_projection\_matrix()} to the conditional distribution $\hPcond{\hrv u=u, \hrv v=v}{\rv x=x, \rv z=z} = \texttt{Psv\_UVgXZ}$.

\lstinputlisting[%
firstline=79,
lastline=84,
]{../counterexample2.m}

We use these symbolic distributions to calculate the intervals $\texttt{Ri} = \ul R^\circ$ and $\texttt{mui} = \ul \mu^\circ$.

\lstinputlisting[%
firstline=120,
lastline=124,
]{../counterexample2.m}

The last section is dedicated to the inner bound. First the rates $\texttt{Rv} = R(i)_{i=1\dots K}$ are calculated and we obtain the grid $\texttt{MUmi} = \ul\mu(i,j)_{i,j=1\dots K}$, which is subsequently shifted to obtain $\ul{\wt{\mu}}$.

\lstinputlisting[%
firstline=137,
lastline=151,
]{../counterexample2.m}

Finally, we use \texttt{upper\_concave\_envelope3\_tri} to obtain the triangularization $\mat T$. Using the function \texttt{tsearchn} we find the triangle that contains $(\ul R^\circ, \ul R^\circ)$. We construct a supporting hyperplane using \texttt{hyperplane3}, verify it with \texttt{verify\_hyperplane} and finally obtain the upper bound to our inner bound $\texttt{MUi} = \ul \mu$ by interpolating on the resulting hyperplane.

\lstinputlisting[%
firstline=168,
]{../counterexample2.m}

Indeed, evaluation confirms that there is a gap with $\ul \mu^\circ - \ul \mu > 10^{-4}$:

\texttt{Distance is at least 1.181411e-04}

\section{Source Files}
\label{sec:source-files}

\begin{itemize}
\item \texttt{utils/binent.m}

  The function $\texttt{binent(x)}$ computes the binary entropy function $\texttt{binent(x)} = \binEnt{x}$. If the argument is a vector, $\binEnt{x}$ it is computed element-wise.

  \lstinputlisting{../utils/binent.m}
\item \texttt{utils/hyperplane3.m}

  Given three interval vectors $\ul{\vt x}, \ul{\vt y}, \ul{\vt z} \subseteq \RR^3$, we obtain the three vectors $\ul{\vt w}_i = (\ul{x}_i, \ul{y}_i, \ul{z}_i)$, with $i \in \{1,2,3\}$.
  Let $\vt w_i$ denote the mean of the interval vector $\ul{\vt w}_i$ for $i \in \{1,2,3\}$. The function the computes a vector $\vt a \perp \{\vt w_1 - \vt w_2, \vt w_1 - \vt w_3\}$ with $a_3 \le 0$. It returns $(\vt a, b) = \texttt{hyperplane3(x,y,z)}$ such that $b$ is maximal with $\vt a \cdot \ul{\vt w}_i \ge b$ for $i \in \{1,2,3\}$.
  
  \lstinputlisting{../utils/hyperplane3.m}
\item \texttt{utils/ent.m}

  This function computes the entropy of a p.m.f.\ $\vt x$, $\texttt{ent(x)} = \binEnt{\vt x}$, where
  \begin{align}
    \binEnt{\vt x} = - \sum_{i=1}^{n} x_i \log_2 x_i .
  \end{align}
  If the argument has more than one dimension, $\binEnt{\vt x}$ is computed along the first dimension.

  \lstinputlisting{../utils/ent.m}
\item \texttt{utils/get\_projection\_matrix.m}

  Let $\AAA \in \RR^{16}$ be the affine space, such that $\AAA \cap [0,1]^{16}$ is the set of conditional probability distributions $\Pcond{\rv u=u, \rv v=v}{\rv x=x, \rv z=z}$ with $x,z,u,v \in \{0,1\}$, such that $\rv u \mkv \rv x \mkv \rv z$, $\rv x \mkv \rv z \mkv \rv v$ and the marginals $\Pcond{\rv u=u}{\rv x=x}$ and $\Pcond{\rv v=v}{\rv z=z}$ are constant.
  We may write $\AAA = \vt x + \AAA'$ where $\vt x \in \AAA$ and $\AAA'$ is a linear space.
  The function $\mat N = \texttt{get\_projection\_matrix()}$ returns a $16 \times 16$ symbolic matrix $\mat N$, that performs an orthogonal projection onto $\AAA'$.
  
  \lstinputlisting{../utils/get_projection_matrix.m}
\item \texttt{utils/ibinent.m}

  The class \texttt{ibinent} provides an interval version of the inverse binary entropy function $\binEntInv{\cdot}$. To construct it, we take advantage of the fact that $\binEnt{\cdot}$ is a concave function.
  
  \lstinputlisting{../utils/ibinent.m}

  
\item \texttt{utils/star.m}

  This functions computes the binary convolution $\texttt{star(a,b)} = a (1-b) + (1-a) b$. If the arguments are vectors, the function is computed element-wise.
  
  \lstinputlisting{../utils/star.m}
  
\item \texttt{utils/upper\_concave\_envelope3\_tri.m}

  Given three vectors $\vt x$, $\vt y$, and $\vt z$, each of length $n$, we interpret them as a points $\vt w_i = (x_i, y_i, z_i)$, where $i \in \Nto{n}$.
  These points span the polytope $\AAA \defas \cvx{\vt w_i : i \in \Nto{n}}$. Let $\OOO \defas \{(x,y,z) : x,y \ge 0, z \le 0 \}$. We are interested in the compact faces of $\AAA' \defas \AAA + \OOO$. The function $\mat T = \texttt{upper\_concave\_envelope3\_tri(x,y,z)}$ returns an $M \times 3$ matrix $\mat T = (\vt k^{(1)}, \vt k^{(2)}, \dots ,\vt k^{(M)})\transp$. The $M$ triangles $\cvx{\vt w_{k^{(m)}_i} : i \in \{1,2,3\}}$, $m \in \Nto{M}$, provide a triangularization of all compact faces of $\AAA'$.
  
  \lstinputlisting{../utils/upper_concave_envelope3_tri.m}
  
\item \texttt{utils/verify\_hyperplane.m}

  This function verifies that the pair $(\vt a, b)$ represents a supporting hyperplane. Given three (interval) vectors $\ul{\vt x}$, $\ul{\vt y}$, and $\ul{\vt z}$, each of length $n$, we interpret them as a (interval) points $\ul{\vt w}_i = (\ul x_i, \ul y_i, \ul z_i)$, where $i \in \Nto{n}$. The function $b=\texttt{verify\_hyperplane(q,x,y,z)}$ returns the Boolean $b = 1$ if $\ul{\vt w}_i \cdot \vt a \ge b$ for all $i \in \Nto{n}$, where $(\vt a, b) = \texttt{q}$. Otherwise the function returns $b = 0$.

  \lstinputlisting{../utils/verify_hyperplane.m}
\item \texttt{utils/z\_interp2.m}

  Given the hyperplane $\HHH \defas \{\vt v \in \RR^3: \vt a \cdot \vt v = b\}$ and two values (possibly intervals) $\ul x$ and $\ul y$, the function $\ul z =  \texttt{z\_interp2(q,x,y)}$ returns $\ul z$ such that $(x,y,z) \in \HHH \cap ( \ul x \times \ul y \times \RR )$ implies $z \in \ul z$. The first argument is $\texttt{q} = (\vt a, b)$.
  
  \lstinputlisting{../utils/z_interp2.m}
\end{itemize}
 
% \begin{appendix}
%   \section{Appendix}
% \end{appendix}


% Bibliography
%\cleardoublepage
\printbibliography[heading=bibintoc]


\end{document}
